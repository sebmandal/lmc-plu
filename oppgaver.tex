\documentclass[11pt, oneside]{article}   	% use "amsart" instead of "article" for AMSLaTeX format
\usepackage{geometry}                		% See geometry.pdf to learn the layout options. There are lots.
\usepackage{asmath}
\geometry{letterpaper}                   		% ... or a4paper or a5paper or ... 
%\geometry{landscape}                		% Activate for rotated page geometry
%\usepackage[parfill]{parskip}    		% Activate to begin paragraphs with an empty line rather than an indent
\usepackage{graphicx}				% Use pdf, png, jpg, or eps§ with pdflatex; use eps in DVI mode
								% TeX will automatically convert eps --> pdf in pdflatex		
\usepackage{amssymb}

%SetFonts

%SetFonts


\title{Oblig 1 - IN1020}
\author{Sebastian Wilhelm Kristian Pritchard-Davies Mandal}
\date{}

\begin{document}
\maketitle

\section*{3 Oppgaven}

\subsection*{3.1 Et register med PLU koder}

Her er fire ekstra egenskaper matvarer kan ha som kan være nyttige å representere digitalt:

\begin{enumerate}
    \item \textbf{Lagerstatus}: Indikerer om varen er på lager (1) eller utsolgt (0).
    \item \textbf{Kategori}: Klassifiserer varen i en gruppe, som "frukt", "grønnsaker", etc. Dette kan være en numerisk kode.
    \item \textbf{Beskrivelse}: En numerisk kode som refererer til en liste med tekstbeskrivelser av varen.
    \item \textbf{Mengde}: Antall enheter kjøpt, nyttig for bulk-kjøp.
\end{enumerate}

Tabellen nedenfor viser hvordan disse egenskapene kan representeres:

\begin{tabular}{|l|l|r|l|l|}
    \hline
    Hva            & Beskrivelse                                          & Antall celler & Verdier      & Datatype \\
    \hline
    PLU            & Koden kassereren slår opp                           & 1             & 0 til 999    & Heltall  \\
    Pris           & Prisen til varen, brukt til å beregne totalpris      & 1             & 0 til 999    & Heltall  \\
    Lagerstatus    & Om varen er på lager (1) eller utsolgt (0)           & 1             & 0 eller 1    & Heltall  \\
    Kategori       & Varenes kategori (numerisk kode)                     & 1             & 0 til 9      & Heltall  \\
    Beskrivelse    & Kode som refererer til en beskrivelse                 & 1             & 0 til 99     & Heltall  \\
    Mengde         & Antall enheter kjøpt                                & 1             & 0 til 999    & Heltall  \\
    \hline
\end{tabular}

\subsection*{3.2 Minnestruktur og Minnebruk}

LMC har 100 minneceller. Vi må planlegge hvordan vi bruker minnet effektivt for både matvaredata og programkode.

\subsubsection*{Minnestruktur for Matvarer}

Hvis hver vare bruker 6 celler, kan vi lagre 2 varer med cellene 0-11:

\begin{tabular}{|r|l|}
    \hline
    Celle & Hva            \\
    \hline
    0-5   & Vare 1 data    \\
    6-11  & Vare 2 data    \\
    \hline
\end{tabular}

Dette betyr at vi kan lagre data for 2 matvarer i cellene 0-11. De resterende cellene kan brukes til programkode og eventuell ekstra data.

\subsubsection*{Bruk av Minne}

\begin{tabular}{|r|l|}
    \hline
    Celleområde  & Hva            \\
    \hline
    0-11         & Data for matvarer (2 varer, hver bruker 6 celler) \\
    12-15        & Kan brukes til ekstra data eller programkode \\
    16-99        & Programkode og eventuelle ekstra data \\
    \hline
\end{tabular}

\subsubsection*{Hvor Mange Matvarer Kan Lagres?}

Med 16 celler til rådighet for matvarer (0-15) og 6 celler per vare, kan du lagre:

\[
\frac{16 \text{ celler}}{6 \text{ celler per vare}} \approx 2 \text{ varer}
\]

\subsubsection*{Oppsummering og Forslag}

1. **Dataområde**:
   - Celler 0-11: Data for 2 matvarer (6 celler per vare).

2. **Kodeområde**:
   - Celler 12-99: Programkode og ekstra data.

For å lagre flere matvarer eller ha plass til mer programkode, kan du vurdere:
- **Effektiv Datarepresentasjon**: Bruk indekser eller kompakte dataformater.
- **Ekstra Minne**: Hvis mulig, se etter måter å utvide minnet eller optimalisere bruken.

\end{document}
